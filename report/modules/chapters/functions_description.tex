\section*{Описание пользовательских функций}

\subsection*{Модуль truck.h}

\subsubsection*{Truck}

\underline{Назначение:} Конструктор для инициализации объекта структуры

\underline{Параметры:} \verb|-|

\underline{Возвращаемое значение:} \verb|Truck|


\subsubsection*{Truck}

\underline{Назначение:} Конструктор для инициализации объекта структуры

\underline{Параметры:} 

\begin{itemize}
    \item \verb|const Truck &truck| -- ссылка на структуру, из которой нужно скопировать поля.
\end{itemize}

\underline{Возвращаемое значение:} \verb|Truck|


\subsection*{Модуль truck\_list.h}


\subsubsection*{TruckList}

\underline{Назначение:} Конструктор для инициализации объекта структуры

\underline{Параметры:} \verb|-|

\underline{Возвращаемое значение:} \verb|TruckList|

\subsubsection*{Get}

\underline{Назначение:} Получение элемента списка

\underline{Параметры:} 

\begin{itemize}
    \item \verb|int index| -- индекс получаемого элемента.
\end{itemize}

\underline{Возвращаемое значение:} \verb|Truck *|


\subsubsection*{Add}

\underline{Назначение:} Добавление элемента в список

\underline{Параметры:} 

\begin{itemize}
    \item \verb|const Truck &truck| -- ссылка на добавляемый элемент.
\end{itemize}

\underline{Возвращаемое значение:} \verb|Truck *|


\subsubsection*{Insert}

\underline{Назначение:} Добавление элемента на определённую позицию в список

\underline{Параметры:} 

\begin{itemize}
    \item \verb|const Truck &truck| -- ссылка на добавляемый элемент.
    \item \verb|int index| -- индекс, перед которым будет вставлен элемент.
\end{itemize}

\underline{Возвращаемое значение:} \verb|Truck *|


\subsubsection*{Delete}

\underline{Назначение:} Удаление элемента из списка

\underline{Параметры:} 

\begin{itemize}
    \item \verb|int index| -- индекс удаляемого элемента.
\end{itemize}

\underline{Возвращаемое значение:} \verb|void|


\subsubsection*{Free}

\underline{Назначение:} Очистка списка

\underline{Параметры:} \verb|-|

\underline{Возвращаемое значение:} \verb|void|


\subsubsection*{\_check\_index}

\underline{Назначение:} Проверка полученного индекса

\underline{Параметры:} 

\begin{itemize}
    \item \verb|int index| -- индекс проверяемого элемента.
\end{itemize}

\underline{Возвращаемое значение:} \verb|void|


\subsection*{Модуль truck\_db.h}


\subsubsection*{PrintAll}

\underline{Назначение:} Вывод списка на экран

\underline{Параметры:} \verb|-|

\underline{Возвращаемое значение:} \verb|void|


\subsubsection*{Print}

\underline{Назначение:} Вывод элемента списка на экран

\underline{Параметры:} 

\begin{itemize}
    \item \verb|int index| -- индекс выводимого элемента.
\end{itemize}

\underline{Возвращаемое значение:} \verb|void|


\subsubsection*{Add}

\underline{Назначение:} Добавление элемента в список

\underline{Параметры:} \verb|-|

\underline{Возвращаемое значение:} \verb|void|


\subsubsection*{Delete}

\underline{Назначение:} Удаление элемента

\underline{Параметры:} 

\begin{itemize}
    \item \verb|int index| -- индекс удаляемого элемента.
\end{itemize}

\underline{Возвращаемое значение:} \verb|void|


\subsubsection*{Edit}

\underline{Назначение:} Редактирование элемента

\underline{Параметры:} 

\begin{itemize}
    \item \verb|int index| -- индекс редактируемого элемента.
\end{itemize}

\underline{Возвращаемое значение:} \verb|void|


\subsubsection*{Insert}

\underline{Назначение:} Вставка элемента

\underline{Параметры:} 

\begin{itemize}
    \item \verb|int index| -- индекс 
    элемента перед которым будет вставлен новый элемент.
\end{itemize}

\underline{Возвращаемое значение:} \verb|void|


\subsubsection*{Clear}

\underline{Назначение:} Очистка списка

\underline{Параметры:} \verb|-|

\underline{Возвращаемое значение:} \verb|void|


\subsubsection*{Load}

\underline{Назначение:} Загрузка списка из файла

\underline{Параметры:} 

\begin{itemize}
    \item \verb|const std::string& db_path| -- путь к файлу
\end{itemize}

\underline{Возвращаемое значение:} \verb|void|


\subsubsection*{Save}

\underline{Назначение:} Сохранение списка в файле

\underline{Параметры:} 

\begin{itemize}
    \item \verb|const std::string& db_path| -- путь к файлу
\end{itemize}

\underline{Возвращаемое значение:} \verb|void|


\subsubsection*{Find}

\underline{Назначение:} Поиск элементов

\underline{Параметры:} 

\begin{itemize}
    \item \verb|bool(*fnd)(Truck* a, const std::string &fld)| -- функция для проверки критериев поиска для одного элемента (на основе лямбда-выражения).
    \item \verb|const std::string &fld| -- искомое значение поля.
\end{itemize}

\underline{Возвращаемое значение:} \verb|void|


\subsubsection*{Sort}

\underline{Назначение:} Сортировка списка

\underline{Параметры:} 

\begin{itemize}
    \item \verb|bool(*gt)(Truck* a, Truck *b)| -- функция сравнения элементов 'greater than' (на основе лямбда-выражения).
\end{itemize}

\underline{Возвращаемое значение:} \verb|void|


\subsubsection*{\_qsort}

\underline{Назначение:} Сортировка списка (QuickSort)

\underline{Параметры:} 

\begin{itemize}
    \item \verb|bool(*gt)(Truck* a, Truck *b)| -- функция сравнения элементов 'greater than'.
    \item \verb|Truck *left| -- левая граница.
    \item \verb|Truck *right| -- правая граница.
\end{itemize}

\underline{Возвращаемое значение:} \verb|void|


\subsubsection*{\_partition}

\underline{Назначение:} Разбиение списка (QuickSort)

\underline{Параметры:} 

\begin{itemize}
    \item \verb|bool(*gt)(Truck* a, Truck *b)| -- функция сравнения элементов 'greater than'.
    \item \verb|Truck *left| -- левая граница.
    \item \verb|Truck *right| -- правая граница.
\end{itemize}

\underline{Возвращаемое значение:} \verb|Truck*|


\subsubsection*{\_swap}

\underline{Назначение:} Перестановка элементов.

\underline{Параметры:} 

\begin{itemize}
    \item \verb|Truck *left| -- первый элемент.
    \item \verb|Truck *right| -- второй элемент.
\end{itemize}

\underline{Возвращаемое значение:} \verb|void|


\subsection*{Модуль cmd.h}


\subsubsection*{Get}

\underline{Назначение:} Получение ввода комманды

\underline{Параметры:} \verb|-|

\underline{Возвращаемое значение:} \verb|Command|


\subsubsection*{Parse}

\underline{Назначение:} Определение команды из ввода

\underline{Параметры:} \verb|-|

\underline{Возвращаемое значение:} \verb|Command|


\subsubsection*{YN}

\underline{Назначение:} Получение подтверждения пользователя (да/нет)

\underline{Параметры:} 

\begin{itemize}
    \item \verb|const std::string &message| -- сообщение
\end{itemize}

\underline{Возвращаемое значение:} \verb|bool|


\subsubsection*{Run}

\underline{Назначение:} Выполнение введённой команды

\underline{Параметры:} 

\begin{itemize}
    \item \verb|TruckDataBase &db| -- ссылка на интерфейс управления списком
\end{itemize}

\underline{Возвращаемое значение:} \verb|void|


\subsubsection*{Add}

\underline{Назначение:} Добавление элемента в список

\underline{Параметры:} \verb|-|

\underline{Возвращаемое значение:} \verb|void|


\subsubsection*{Load}

\underline{Назначение:} Загрузка списка из файла

\underline{Параметры:} \verb|-|

\underline{Возвращаемое значение:} \verb|void|


\subsubsection*{Save}

\underline{Назначение:} Сохранение списка в файле

\underline{Параметры:} \verb|-|

\underline{Возвращаемое значение:} \verb|void|


\subsubsection*{Print}

\underline{Назначение:} Вывод элемента на экран по индексу

\underline{Параметры:} \verb|-|

\underline{Возвращаемое значение:} \verb|void|


\subsubsection*{PrintAll}

\underline{Назначение:} Вывод всех элементов на экран

\underline{Параметры:} \verb|-|

\underline{Возвращаемое значение:} \verb|void|


\subsubsection*{Help}

\underline{Назначение:} Вывод справки

\underline{Параметры:} \verb|-|

\underline{Возвращаемое значение:} \verb|void|


\subsubsection*{Insert}

\underline{Назначение:} Вставка элемента

\underline{Параметры:} \verb|-|

\underline{Возвращаемое значение:} \verb|void|


\subsubsection*{Delete}

\underline{Назначение:} Удаление элемента

\underline{Параметры:} \verb|-|

\underline{Возвращаемое значение:} \verb|void|


\subsubsection*{Edit}

\underline{Назначение:} Редактирование элемента

\underline{Параметры:} \verb|-|

\underline{Возвращаемое значение:} \verb|void|


\subsubsection*{SortByBrand}

\underline{Назначение:} Сортировка списка по марке

\underline{Параметры:} \verb|-|

\underline{Возвращаемое значение:} \verb|void|


\subsubsection*{SortByCapacity}

\underline{Назначение:} Сортировка списка по грузоподъёмности

\underline{Параметры:} \verb|-|

\underline{Возвращаемое значение:} \verb|void|


\subsubsection*{SortByDistance}

\underline{Назначение:} Сортировка списка по дальности перевозки

\underline{Параметры:} \verb|-|

\underline{Возвращаемое значение:} \verb|void|


\subsubsection*{FindByBrand}

\underline{Назначение:} Поиск элементов по марке

\underline{Параметры:} \verb|-|

\underline{Возвращаемое значение:} \verb|void|


\subsubsection*{FindByCapacity}

\underline{Назначение:} Поиск элементов по грузоподъёмности

\underline{Параметры:} \verb|-|

\underline{Возвращаемое значение:} \verb|void|


\subsubsection*{FindByDistance}

\underline{Назначение:} Поиск элементов по дальности перевозки

\underline{Параметры:} \verb|-|

\underline{Возвращаемое значение:} \verb|void|


\subsection*{Модуль input.h}

\subsubsection*{Input}

\underline{Назначение:} Ввод данных типа \verb|int|

\underline{Параметры:} 

\begin{itemize}
	\item \verb|int &element| -- ссылка на вводимый элемент
	\item \verb|const std::string &message = ""| -- сообщение, приглашающее к вводу
	\item \verb|int l = _min_int| -- левая граница ввода
	\item \verb|int r = _max_int| -- правая граница ввода
\end{itemize}

\underline{Возвращаемое значение:} \verb|int|


\subsubsection*{Input}

\underline{Назначение:} Ввод данных типа \verb|float|

\underline{Параметры:} 

\begin{itemize}
    \item \verb|float &element| -- ссылка на вводимый элемент
    \item \verb|const std::string &message = ""| -- сообщение, приглашающее к вводу
    \item \verb|float l = _min_float| -- левая граница ввода
    \item \verb|float r = _max_float| -- правая граница ввода
\end{itemize}

\underline{Возвращаемое значение:} \verb|float|


\subsubsection*{Input}

\underline{Назначение:} Ввод данных типа \verb|std::string|

\underline{Параметры:} 

\begin{itemize}
    \item \verb|std::string &element| -- ссылка на вводимый элемент
    \item \verb|const std::string &message = ""| -- сообщение, приглашающее к вводу
\end{itemize}

\underline{Возвращаемое значение:} \verb|std::string|